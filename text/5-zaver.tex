\chapter{Závěr}
\label{5-zaver}

Hlavním cílem této bakalářské práce byla implementace zásuvného modulu
do QGIS rozšiřujícího softwarové vybavení Státního ústavu pro radiační
ochranu v.v.i. (i~jiných institucí). Nástroj poslouží jako další
prostředek k~zrychlení reakční doby při cvičeních radiačních havárií
příp. při samotných radiačních haváriích a především k~zvýšení
bezpečnosti mobilních skupin pracujících přímo v~terénu. Dále si práce
kladla za~cíl seznámit čtenáře s~dopady ionizujícího záření na lidský
organismus a s~metodou pozemního snímání radiace.

Během pozemního monitorování dochází
k~absorbaci energie ionizujícího záření pracovníky mobilních skupin. 
Legislativou jsou
stanoveny hodnoty, které množství absorbované dávky nesmí
překročit. V~závislosti na bezpečnosti práce je tedy třeba plánovat
trasy monitorování tak, aby pracovníci mobilních skupin nebyli vystaveni zdraví
ohrožujícím dávkám.

Vytvářený softwarový nástroj má sloužit jako pomůcka k~plánování bezpečných tras
monitorování. Dle výsledků výpočtu nástroje, tedy mimo jiné
nakumulovaných dávek v~jednotlivých bodech trasy, může krizový štáb
upravit trasu, případně mo\-difikovat doporučenou rychlost vozidla tak,
aby se dlouho nezdržovalo v~simulaci vytipovaných \uv{horkých}
místech. Jelikož hodnoty vstupující do výpočtu jsou interpolované a
nejsou tedy exaktní, nástroj také neposkytuje exaktní výsledky a~není
možné ho použít pro důležitá rozhodování, např. o~evakuaci
obyvatelstva atp. Dalším využitím nástroje může být například
plánování tras evakuace obyvatel z~vybraných oblastí při havarijním
plánování nebo cvičení. Nástroj byl vytvořen na objednávku Státního
ústavu radiační ochrany, v.v.i.

Nástroj byl s~pomocí návrhů a~požadavků na~úpravy od~pracovníků
\zk{SÚRO} vytvořen na~míru tak, aby splnil očekávání a~byl zároveň 
co~nejvíce intuitivní. Byl napsán v~anglickém jazyce s~vidinou
jeho využití i~v~jiných než českých institucích. Zároveň byl
v~angličtině vytvořen návod na~jeho používání.

\section{Licence a dostupnost zásuvného modulu} Zásuvný modul byl
vytvořen pod licencí \zk{GNU GPL}, kterou podědil po QGIS knihovnách,
jež byly k~implementaci využity. Zásuvný modul je volně dostupný z~CTU
GeoForAll Lab QGIS
%%% ML: tady mate odkaz na online manual
%%% ML: mluvite o Git anebo QGIS repozitari (zde by mel byt odkaz na git)
repozitáře\footnote{\texttt{https://github.com/ctu-geoforall-lab-projects/bp-kala-2017}}. V~budoucnu
je dále v~plánu začlenění zásuvného modulu do oficiálního QGIS
repozitáře.

 



