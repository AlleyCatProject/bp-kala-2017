\chapter{Závěr}
\label{5-zaver}

Hlavním cílem této bakalářské práce byla implementace zásuvného modulu
do QGIS rozšiřujícího softwarové vybavení Státního ústavu pro radiační
ochranu v.v.i. (i~jiných institucí). Nástroj poslouží jako další
prostředek k~zrychlení reakční doby při cvičeních radiačních havárií
příp. při samotných radiačních haváriích a především k~zvýšení
bezpečnosti mobilních skupin pracujících přímo v~terénu. Dále si práce
kladla za cíl seznámit čtenáře s~dopady ionizujícího záření na lidský
organismus a s~metodou pozemního snímání radiace.

Vytvářený nástroj ze vstupní interpolované mapy dávkových příkonů gama
záření vyextrahuje data na dané trase a spočte několik základních
statistik. Zejména se jedná o~celkovou dávku gama záření, kterou
mobilní skupina na trase obdrží. V~případě překročení mezních hodnot
obdržené dávky, které již ohrožují na zdraví, je možné pomocí dalších
výsledků výpočtu vytipovat místa na trase, kterým se vyhnout a tedy
trasu monitorování modifikovat.

Nástroj byl s~pomocí návrhů a požadavků na úpravy od pracovníků
\zk{SÚRO} vytvořen na míru tak, aby splnil očekávání a byl zároveň co
nejvíce intuitivní. Nástroj byl vytvořen v~anglickém jazyce s~vidinou
jeho využití i v~jiných než českých institucích. Zároveň byl
v~angličtině vytvořen návod na jeho používání.


\section{Licence a dostupnost zásuvného modulu} Zásuvný modul byl
vytvořen pod licencí \zk{GNU GPL}, kterou podědil po QGIS knihovnách,
jež byly k~implementaci využity. Zásuvný modul je volně dostupný z~CTU
GeoForAll Lab QGIS
%%% ML: tady mate odkaz na online manual
%%% ML: mluvite o Git anebo QGIS repozitari (zde by mel byt odkaz na git)
repozitáře\footnote{\texttt{https://ctu-geoforall-lab-projects.github.io/bp-kala-2017/}}. V~budoucnu
je dále v~plánu začlenění zásuvného modulu do oficiálního QGIS
repozitáře.

 



