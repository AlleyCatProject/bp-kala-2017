\chapter{Teoretický základ}
\label{2-teorie}

Tato kapitola se zabývá..
\section{Ionizující záření}

\section{Jaderná bezpečnost}
Bezpečnostní opatření, která nemusí být potřeba, ale měly by být připravené.

\section{Radiační ochrana}
Co to je
\subsection{Historie}
https://www.suro.cz/cz/radiacni-ochrana/historie
\subsection{Biologické účinky záření}
%%https://www.fbi.vsb.cz/export/sites/fbi/050/.content/sys-cs/resource/PDF/studijni-materialy/zareni.pdf

\subsection{Monitorování radiace}
Jak se to měří

\section{Fyzikální základ}

Jednotky, převody...




%%asim.utia.cas.cz/reporty/2011/HZS_FINAL_po%20zkraceni.pdf

%%sujb.cz/fileadmin/sujb/docs/zpravy/vyrocni_zpravy/ceske/VZ_SUJB_2015_FIN_cast_II.pdf

%%sujb.cz/aktualne/detail/clanek/zona-2015-cviceni-simulovane-havarie-v-temeline-1

Zeptat se, čím SÚRO monitoruje pozemní radiaci (jakým vehiklem, nějak ho popsat)

https://www.suro.cz/cz/vyzkum/vysledky/metodiky/Metodika%20detekce%20radioaktivnich%20latek%20na%20zasazenem%20uzemi.pdf/view

https://www.suro.cz/cz/vyzkum/vysledky/mobilni-a-stacionarni-radiacni-monitorovaci-systemy-nove-generace-pro-radiacni-monitorovaci-site-mostar

Mobilní a stacionární radiační monitorovací systémy nové generace pro radiační monitorovací sítě (MOSTAR)

Metody radiační ochrany, upřesnit k čemu se to bude používat, nerad bych tam psal nesmysly.